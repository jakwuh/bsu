\documentclass[12pt,a4paper]{article}
\usepackage[utf8]{inputenc}
\usepackage[left=1.5cm,right=1.5cm,
top=1.5cm,bottom=1.5cm,bindingoffset=0cm]{geometry}
\usepackage[english,russian]{babel}
\usepackage[pdftex]{graphicx}
\usepackage{amsfonts}
\usepackage{amsmath}
\usepackage{verbatim}
\begin{document}
	\begin{titlepage}
	
		\centerline{\large \bf Белорусский государственный университет}
		\centerline{\large \bf Факультет прикладной математики и информатики}
		\vfill
		\vfill
		\centerline{\Large \bf Построение}
		\centerline{\Large \bf интерполяционного многочлена Лагранжа}
		\bigskip
		\vfill
		\bigskip
		\vfill
		\begin{centering}
			{\large
				Отчет по лабораторной работе \\
				студентки 2 курса 3 группы \\
				Бурак Ирины \\
			}
		\end{centering}
		\vfill
		\vfill
		\hfill
		\begin{minipage}{0.25\textwidth}
			{\large{\bf Преподаватель} \\
				{Будник А.М.}}
		\end{minipage}
		\vfill
		\vfill
		\centerline{\Large \bf Минск 2015}
	\end{titlepage}
	\section{Постановка задачи}
	Для некоторой функции $f(x)$ известны её значения $f(x_i)$ в точках $x_i, i = \overline{0, n}$. Требуется восстановить функцию и в других точках заданного отрезка $[a, b]$. 
	\section{Построение многочлена Лагранжа}
	Многочлен в форме Лагранжа записывается в виде:
	$$P_n(x) = \sum_{i = 0}^{n}\frac{\omega(x)}{(x - x_i)\omega'(x_i)}*f(x_i),$$
	где $\omega(x) = (x - x_0)(x - x_1)\dots(x - x_n)$, а $\omega'(x_i) = (x_i - x_0)\dots(x - x_{i-1})(x - x_{i+1})\dots(x - x_n)$.
	\section{Оценка погрешности}
	Предполагается,что функция $f(x)$ является достаточное число раз дифференцируемой. Тогда, если $\forall x \in[a, b] \leftarrow |f^{(n+1)}(x)| \le M = const$, то верна следующая оценка погрешности:
	$$|f(x) - P_n(x)| = \frac{M}{(n+1)!}|\omega(x)|$$
	\section{Реализация}
	Для функции $f(x) = 0.9e^x + 0.1e^{-x}, [a, b] = [0, 2]$ получаем следующие таблицы значений:
	\begin{itemize}
		\item $n = 5:$ \\
		\includegraphics[scale = 0.6]{Table_5.png}
		\item $n = 10:$ \\
		\includegraphics[scale = 0.6]{Table_10.png}
	\end{itemize}
	Для оценки погрешности получаем: 
	$$|f^{n+1}(x)| = |0.9e^{x} + (-1)^{n+1}*0.1e^{-x}| \le 0.9e^{x} + 0.1e^{-x} \le 0.9e^2 + 0.1 = M$$
	\section{Листинг}
	\verbatiminput{Listing.txt}
	\section{Результаты}
	Для оценки точности результата были рассмотрены точки в начале, середине и конце отрезка:
	\begin{itemize}
		\item $n = 5:$ \\
		\includegraphics[scale = 0.8]{Results_5.png}
		\item $n = 10:$ \\
		\includegraphics[scale = 0.8]{Results_10.png}
	\end{itemize}
	Исходя из величин погрешностей можно сделать вывод, что при увеличении $n$ многочлен Лагранжа точнее приближает функцию. \\
	Наименьшую погрешность в данном случае приближение даёт в середине рассматриваемого отрезка. \\
	Теоретическая величина погрешности в силу нестрогих оценок оказывается на практике завышенной. 
\end{document}