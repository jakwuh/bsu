\documentclass[10pt,a4paper]{article}
\usepackage[utf8]{inputenc}
\usepackage[left=1cm,right=1cm,
top=1cm,bottom=1cm,bindingoffset=0cm]{geometry}
\pagestyle{empty}
\usepackage[english,russian]{babel}
\usepackage[pdftex]{graphicx}
\usepackage{amsfonts}
\usepackage{amsmath}
\usepackage{verbatim}
\begin{document}
	\begin{titlepage}
		
		\centerline{\large \bf Белорусский государственный университет}
		\centerline{\large \bf Факультет прикладной математики и информатики}
		\vfill
		\vfill
		\vfill
		\vfill
		\vfill
		\vfill
		\centerline{\Large \bf Джеймс Аквух}
		\centerline{\Large \bf 3 группа}
		\bigskip
		\vfill
		\bigskip
		\vfill
		\vfill
		\vfill
		\hfill
		\vfill
		\vfill
		\centerline{\Large \bf Минск 2016}
	\end{titlepage}
	\noindent
	$u'(x)=c'(x)e^{-\sqrt{1-x}}+c(x)$\\ \\
	$c'(x)=e^{\sqrt{1-x}}$\\ \\
	$u(x)=ce^{-\sqrt{1-x}}-2\sqrt{1-x}+2$\\ \\
	$u(0)=0 \Rightarrow c=0$\\ \\
	\textbf{Ответ:} \\ \\
	$y(x)=\sqrt{1+x}(1-\sqrt{1-x})+\sqrt{1-x^2}=\sqrt{1+x}$\\ \\
	
	\noindent\textbf{1.1} \\  \\
	Определить, является ли нормой в $|x(0)|+|x'(0)|+\max\limits_{t\in {[0,1]}} |x''(t)|$. \\ \\
	\textbf{Решение.} \\ \\
	Первая аксиома:
	$ ||x|| = 0 \Leftrightarrow x = 0$. Очевидно. \\
	Вторая аксиома:
	$ ||\alpha x|| = |\alpha|||x||$. Очевидно.\\
	Третья аксиома:
	$||x+y||<=|x|+|y|$. Очевидно.\\
	\textbf{Ответ: является нормой.} \\ \\
	
	\noindent\textbf{2.1} \\  \\
	Найти предел в $C[a, b]$, если он существует.\\ \\
	$x_n(t) = t\arctan(nt), t \in [0, 2]$. \\ \\
	\textbf{Решение.} \\ \\
	Предельная функция:
	$x_0(t) = \lim\limits_{n \rightarrow \infty} x_n(t) = \lim\limits_{n \rightarrow \infty} t\arctan(nt) = t \frac{\pi}{2}$\\
	$||x_0(t)-x_n(t)||=\max\limits_{0<=t<=3} |t\arctan(tn) - t\frac{\pi}{2}|<\epsilon$\\
	$\forall t \in [0,3], \epsilon > 0 \exists n_0 = \frac{\tan(\frac{\pi}{2}-\frac{\epsilon}{2})}{t} : \forall n>n_0 \Rightarrow |t\arctan(tn) - t\frac{\pi}{2}|<\epsilon$\\
	Следовательно, последовательность сходится.\\
	\textbf{Ответ: $t\frac{\pi}{2}$.} \\ \\
	
	
	\noindent\textbf{3.1} \\  \\
	Найти предел последовательности в пространстве $l_{3/2}$, если он существует. \\
	$$x_n = \left(\left(\frac{5n + 1}{5n + 2}\right)^n, \dots, \left(\frac{5n + 1}{5n + 2}\right)^n, \dots \right) $$
	\textbf{Решение.} \\ \\
	$$\lim\limits_{n \leftarrow \infty}\left(\frac{5n + 1}{5n + 2}\right)^n = \dots = \frac{1}{\sqrt[5]{e}}$$
	$$x_0 = \left(\frac{1}{\sqrt[5]{e}}, \frac{1}{\sqrt[5]{e}}, \dots \right) $$
	$$\sum_{n = 1}^{\infty} \frac{1}{\sqrt[5]{e}} = \infty \Rightarrow x_0 \notin l_{3/2}$$
	\textbf{Ответ: предела нет.} \\ \\
\end{document}